\documentclass[a4paper, 12pt]{scrartcl}
  % Headers and footers.
  \usepackage{fancyhdr}
  % Enables the use of colour.
  \usepackage{xcolor}
  % Syntax high-lighting for code. Requires Python's pygments.
  \usepackage{minted}
  % Enables the use of umlauts and other accents.
  \usepackage[utf8]{inputenc}
  % Diagrams.
  \usepackage{tikz}
  % Settings for captions, such as sideways captions.
  \usepackage{caption}
  % Symbols for units, like degrees and ohms.
  \usepackage{gensymb}
  % Latin modern fonts - better looking than the defaults.
  \usepackage{lmodern}
  % Allows for columns spanning multiple rows in tables.
  \usepackage{multirow}
  % Better looking tables, including nicer borders.
  \usepackage{booktabs}
  % More math symbols.
  \usepackage{amssymb}
  % More math fonts, like \mathbb.
  \usepackage{amsfonts}
  % More math layouts, equation arrays, etc.
  \usepackage{amsmath}
  % More theorem environments.
  \usepackage{amsthm}
  % More column formats for tables.
  \usepackage{array}
  % Adjust the sizes of box environments.
  \usepackage{adjustbox}
  % Better looking single quotes in verbatim and minted environments.
  \usepackage{upquote}
  % Better blank space decisions.
  \usepackage{xspace}
  % Better looking tikz trees.
  \usepackage{forest}
  % URLs.
  \usepackage[hidelinks]{hyperref}
  % Plotting.
  \usepackage{pgfplots}
  % Calculates the number of pages.
  \usepackage{lastpage}
  % Styling the abstract.
  \usepackage{abstract}

  
  % Various tikz libraries.
  % For drawing mind maps.
  \usetikzlibrary{mindmap}
  % For adding shadows.
  \usetikzlibrary{shadows}
  % Extra arrows tips.
  \usetikzlibrary{arrows.meta}
  % Old arrows.
  \usetikzlibrary{arrows}
  % Automata.
  \usetikzlibrary{automata}
  % For more positioning options.
  \usetikzlibrary{positioning}
  % Creating chains of nodes on a line.
  \usetikzlibrary{chains}
  % Fitting node to contain set of coordinates.
  \usetikzlibrary{fit}
  % Extra shapes for drawing.
  \usetikzlibrary{shapes}
  % For markings on paths.
  \usetikzlibrary{decorations.markings}
  % For advanced calculations.
  \usetikzlibrary{calc}
  
  % GMIT colours.
  \definecolor{gmitblue}{RGB}{20,134,225}
  \definecolor{gmitred}{RGB}{220,20,60}
  \definecolor{gmitgrey}{RGB}{67,67,67}

  % Set minted up.
  \usemintedstyle{manni}
  \setminted{baselinestretch=1.2, bgcolor=gmitgrey!10}

  % Change the displayed name of the references section.
  \renewcommand{\refname}{\selectfont\large References} 

  % Heading
  \title{\vspace{-20mm}Tasks 2020}
  \author{Fundamentals of Data Analysis}
  \date{Due: last commit on or before December 18\textsuperscript{th}, 2020\vspace{-6mm}}

\begin{document}
  
  \maketitle

  These are the instructions for the Tasks assessment for Fundamentals of Data Analysis in 2020.
  The assessment is worth 50\% of the marks for the module.
  Please read the \emph{Using git for assessments}~\cite{usinggit} document on the Moodle page which applies here.
  As always, you must also follow the code of student conduct and the policy on plagiarism~\cite{gmitqaf}.

  \section*{Instructions}
    Four tasks will be listed here at different times during the semester.
    You should complete all tasks in a single jupyter notebook.
    This, along with relevant files like a README, should be in a single git repository synced with a hosting provider like GitHub~\cite{github}.
    That URL should then be submitted using the link on the Moodle page.

    \begin{enumerate}
      \item \textbf{October 5\textsuperscript{th}, 2020:}
        Write a Python function called \mintinline{python}{counts} that takes a list as input and returns a dictionary of unique items in the list as keys and the number of times each item appears as values.
        So, the input \mintinline{python}{['A', 'A', 'B', 'C', 'A']} should have output \mintinline{python}{{'A': 3, 'B': 1, 'C': 1}}.
        Your code should not depend on any module from the standard library\footnote{By the standard library, we mean the modules and packages that come as standard with Python. Anything built-in that can be used without an \mintinline{python}{import} statement can be used.} or otherwise.
        You should research the task first and include a description with references of your algorithm in the notebook.
      \item \textbf{November 2\textsuperscript{nd}, 2020:}
        Write a Python function called \mintinline{python}{dicerolls} that simulates rolling dice.
        Your function should take two parameters: the number of dice $k$ and the number of times to roll the dice $n$.
        The function should simulate randomly rolling $k$ dice $n$ times, keeping track of each total face value.
        It should then return a dictionary with the number of times each possible total face value occurred.
        So, calling the function as \mintinline{python}{diceroll(k=2, n=1000)} should return a dictionary like:
        \begin{minted}{python}
{2:19,3:50,4:82,5:112,6:135,7:174,8:133,9:114,10:75,11:70,12:36}
        \end{minted}
        You can use any module from the Python standard library you wish and you should include a description with references of your algorithm in the notebook.
        \newpage
      \item \textbf{November 16\textsuperscript{th}, 2020:}
        The \mintinline{python}{numpy.random.binomial} function can be used to simulate flipping a coin with a $50/50$ chance of heads or tails.
        Interestingly, if a coin is flipped many times then the number of heads is well approximated by a bell-shaped curve.
        For instance, if we flip a coin 100 times in a row the chance of getting 50 heads is relatively high, the chances of getting 0 or 100 heads is relatively low, and the chances of getting any other number of heads decreases as you move away from 50 in either direction towards 0 or 100.
        Write some python code that simulates flipping a coin 100 times.
        Then run this code 1,000 times, keeping track of the number of heads in each of the 1,000 simulations.
        Select an appropriate plot to depict the resulting list of 1,000 numbers, showing that it roughly follows a bell-shaped curve.
        You should explain your work in a Markdown cell above the code.
      \item \textbf{November 30\textsuperscript{th}, 2020:}
        Simpson's paradox is a well-known statistical paradox where a trend evident in a number of groups reverses when the groups are combined into one big data set.
        Use numpy to create four data sets, each with an \mintinline{python}{x} array and a corresponding \mintinline{python}{y} array, to demonstrate Simpson's paradox.
        You might create your \mintinline{python}{x} arrays using \mintinline{python}{numpy.linspace} and create the \mintinline{python}{y} array for each \mintinline{python}{x} using notation like \mintinline{python}{y = a * x + b} where you choose the \mintinline{python}{a} and \mintinline{python}{b} for each \mintinline{python}{x}, \mintinline{python}{y} pair to demonstrate the paradox.
        You might see the Wikipedia page for Simpson's paradox for inspiration.

      \end{enumerate}


  \newpage

  \section*{Marking scheme}
    The following marking scheme will be used to mark your submission out of 100\%, which will then be scaled to 50\%.
    The examiner's overall impression of your submission may influence marks in each individual component.
    It is important that your submission provides direct evidence of each of the items listed in each category.
    For instance, your commit history should demonstrate and provide evidence that you had a pragmatic attitude to completing the assessment.
    Likewise, your submission should have references in it to demonstrate that you considered the literature and the work of others.

    \begin{center}
      \begin{tabular}{llp{8.4cm}}
        \toprule
        25\% & \textbf{Research} & Evidence of research performed on topic; submission based on referenced literature, particularly academic literature; evidence of understanding of the documentation for any software or libraries used. \\
        \midrule
        25\% & \textbf{Development} & Environment can be set up as described; code works without tweaking and as described; code is efficient, clean, and clear; evidence of consideration of standards and conventions appropriate to code of this kind. \\
        \midrule
        25\% & \textbf{Consistency} & Evidence of planning and project management; pragmatic attitude to work as evidenced by well-considered commit history; commits are of a reasonable size; consideration of how commit history will be perceived by others. \\
        \midrule
        25\% & \textbf{Documentation} & Clear documentation of how to create an environment in which any code will run, how to prepare the code for running, how to run the code including setting any options or flags, and what to expect upon running the code. Concise descriptions of code in comments and README. \\
        \bottomrule
      \end{tabular}
    \end{center}
  
  \bibliographystyle{ieeetrans}
  \bibliography{bibliography}

\end{document}
